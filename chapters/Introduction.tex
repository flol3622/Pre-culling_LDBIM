\chapter{Introduction}
\section{Context}
\subsection{3D viewers}
-> Applications?

-> Who uses them?

-> What for?

\subsection{\acs{bim} geometry}
-> What is \ac{bim}? (short)

-> Extend of \ac{bim} geometry?

-> Complexity of \ac{bim} geometry?

\subsection{\acs{ldbim}}
-> ! Focus on geometry!

-> What is \ac{ldbim}?

-> Why the need / What are the advantages of \ac{ldbim}?

-> Context od enrichment and complexity

\subsection{Computing power dilemma}
-> What is the hardware problem?

-> Why is it that important for the \ac{aec} industry?

->


\section{Research questions}

-> Why the need for this thesis? (why a \ac{ldbim} viewer?)

-> What is the possible solution? (Culling algorithms)

-> Why the need for research questions?
(culling algorithms are not new, always progress, see later)

\subsection[Can \acs{ldbim} be culled?]{To which extent can \acs{ldbim} geometry be culled\\
    to be streamed to lightweight viewers?}
-> What can be culled exactly?

-> What needs to be streamed?

-> What is the impact of culling on the viewing experience?

\subsection[Can existing semantic be used?]{Can existing semantic and ontologies be used to\\
    feed possible culling algorithms?}
-> What are ontologies?

-> Can GIS ontologies be used too?

-> What are the advantages of using ontologies?

\section{Research objectives}
\subsection[Advantages of LDBIM]{Bring forward the advantages of \acs{ldbim} for visualization of big 3D models}
-> Showcase that existing models are already mature enough for these usecases.

-> 
\subsection[Showcase the fisability]{Showcase the fisability of \acs{ldbim} for visualization of big 3D models}
-> 