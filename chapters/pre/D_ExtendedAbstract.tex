\selectlanguage{english}
\begin{center}
    \sffamily
    \huge Extended Abstract

    \Large Philippe Soubrier

    \normalsize
    Supervisor: Prof.\ ir.-arch.\ Paulus Present\\
    Counselors: Ir.-arch.\ Jeroen Werbrouck, Prof.\ dr.\ ir.\ arch.\ Ruben Verstraeten
\end{center}
\begin{refsection}
    \defbibheading{bibliography}[]{}
    \begin{multicols}{2}
        \small
        \titlespacing\subsection{0pt}{5pt}{2pt}
        \emph{Abstract} \textbf{
            - This thesis tackles the issue of visualizing complex \ac{bim} models on resource-limited devices in the \ac{aec} sector, introducing \enquote{pre-culling} to reduce the 3D scene to the viewer's view frustum. By utilizing Semantic Web technologies like the \ac{rdf} and the \ac{sparql}, \ac{bim} models are stored and queried as databases, thereby facilitating their segmentation.
        }

        \emph{Keywords} \textbf{
            - Linked Data, lightweight viewers, culling, BIM, SPARQL
        }

        The \ac{aec} industry has undergone significant technological transformations over the past few decades. The advent of 3D modeling constituted a major shift in the industry, creating a new era of geometric data representation. The subsequent introduction of \ac{bim} has functioned as a versatile repository for semantics derived from various applications throughout the design and construction processes, including but not limited to: cost estimation, energy analysis, and \ac{fm}. As \cite{Werbrouck2018} highlighted, the next challenge for the \ac{aec} industry is related to the domain-specific nature of current \ac{bim} software solutions, which remain inaccessible to other disciplines. This data \textbf{management} issue is currently being tackled by entities such as the \ac{lbd-cg} and academic institutions like the University of Ghent, leveraging the potential of Semantic Web technologies for more integrated and efficient data handling, thereby enabling interdisciplinary collaboration. The term \ac{ldbim} is used in this thesis to denote this emerging field.

        Alongside this evolution, in addition to the increasing amount of semantic data, the geometric data describing a \ac{bim} model is growing in size and complexity. This makes the task of visualization increasingly complex on newer, resource-limited devices used in the industry. Such devices, smartphones and tablets for example, are becoming increasingly popular on construction sites. Existing solutions use a filtering step in the visualization process, known as culling. This computational step necessitates the processing of the entire 3D file or scene to cull it to the viewer's view frustum \parencite{Johansson2015}. This operation produces a minimal 3D scene, which is then utilized for the significantly more resource-intensive process of rendering.

        A bottleneck arises when the initial 3D scene, upon which the in-viewer culling process is performed, becomes too large for the device to handle. As a solution to this problem, this thesis proposes the use of a \textbf{pre-culling} step. In this process, with the aid of Semantic Web technologies, a \ac{bim} model is stored and queried as a partitioned database. By segregating the storage of the \ac{bim} model, offloading the computationally intensive task of culling to a storage server in a preliminary culling step, the amount of data that needs to be processed by the visualization device is minimized.

        This thesis will investigate the extent to which \ac{ldbim} geometry can be pre-culled. It will examine the minimum size of the geometry that can be defined and queried in a \ac{rdf} database. For this purpose, existing ontologies in the field of \ac{aec}, or related fields such as \ac{gis} will be explored. These ontologies will also be scrutinized for their potential utilization within culling algorithms, similar to the work of \cite{Johansson2009}. In their paper, they showcase the use of semantics of a \ac{bim} model in an \ac{ifc} format to feed culling algorithms. \cite{Johansson2015} have also concluded that the inherent underlying hierarchy of \ac{bvh} inside \ac{bim} models could be used to enhance the performance of in-viewer culling algorithms, such as the \ac{chc}++.

        \subsection*{Outline}
        \textsf{Chapter \ref{ch:introduction} -}
        This introduction chapter presents the motivation for this thesis and the research questions. It also introduces the concept of culling.

        \textsf{Chapter \ref{ch:linkedData} -}
        As an introduction to the Semantic Web, this chapter presents the Linked Data principles used in this thesis. Furthermore, it explains the complexity and size of \ac{bim} models within Semantic Web databases.

        \textsf{Chapter \ref{ch:stateOfTheArt} -}
        \cite{Johansson2015} pointed out that there is a scarcity of research exploring the performance of current \ac{bim} viewers. Thus, this state-of-the-art chapter aims to concentrate on the overall features of some promising newer viewers and the ontologies/tools that will be used in this thesis.

        \textsf{Chapter \ref{ch:dynamicQueries} -}
        The concept of dynamic queries is introduced in this chapter. These represent the automatic generation of queries responsible for feeding culling algorithms, to obtain the data needed to visualize the scene visible to the viewer. Firstly, the needs of an \ac{ldbim} viewer are explored. Secondly, three approaches where the culling action is performed are presented: in the query, in the viewer, and in situ. Each approach showcases the possibilities of culling by constructing \ac{sparql} queries.

        \textsf{Chapter \ref{ch:modularApproach} -}
        In addition to the actual culling, this chapter proposes a modular approach to implement the culling process in an \ac{ldbim} viewer. This robust foundation serves as a framework for future research. Informed by this thesis' prototype, the framework identifies the different components needed to create an \ac{ldbim} viewer, and integrates the culling process into it. The chapter dissects the framework into four main components: the viewer itself, the cache manager, the query processor, and the data fetcher.

        \textsf{Chapter \ref{ch:prototype} -}
        To demonstrate the feasibility of the proposed culling process, a prototype is developed in this chapter. This prototype is a proof of concept that implements the modular approach, presented in the previous chapter, within a web-based viewer. It showcases the feasibility of the proposed dynamic queries and also demonstrates the capacity for expansion, such as the visualisation of related semantics.

        \subsection*{Conclusion}
        \textsf{Chapter \ref{ch:conclusion} -} This thesis has demonstrated the feasibility of using culling techniques on \ac{ldbim} graphs. This process effectively reduces the scene size that a lightweight viewer must handle for visualising a large building model stored in a database using Semantic Web technologies. It also presented multiple approaches to perform the culling, each with its own advantages and disadvantages, as well as a modular approach to implement the whole process in a web viewer. This technology has proven to be a viable solution to the demanding needs of the \ac{aec} industry when visualizing large \ac{bim} models. And it provides a solid foundation for further development in various use cases and scenarios requiring 3D visual representation.

        \subsection*{References}
        {\renewcommand*{\bibfont}{\small}
            \printbibliography}
        {\footnotesize
            \textsf{Demonstration:}\\
            \url{https://github.com/flol3622/Pre-culling_LDBIM#demo}\\
            \textsf{Prototype:}\\
            \url{https://github.com/flol3622/LDBIM-viewer}}


    \end{multicols}

\end{refsection}