\newgeometry{bottom=2cm,top=2cm,left=1.5cm,right=1.5cm}
\begin{center}
    \sffamily
    \huge Extended Abstract

    \Large Philippe Soubrier

    \normalsize
    Supervisor: Prof.\ ir.-arch.\ Paulus Present\\
    Counselors: Ir.-arch.\ Jeroen Werbrouck, Prof.\ dr.\ ir.\ arch.\ Ruben Verstraeten
\end{center}
\begin{refsection}
    \defbibheading{bibliography}[]{}
    \begin{multicols}{2}
        \small
        \emph{Abstract} \textbf{
            - This thesis tackles the issue of visualizing complex \ac{bim} models on resource-limited devices in the \ac{aec} sector, introducing \enquote{pre-culling} to reduce the 3D scene to the viewer's view frustum. By utilizing Semantic Web technologies like the \ac{rdf} and the \ac{sparql}, \ac{bim} models are stored and queried as databases, thereby facilitating their segmentation.
        }

        \emph{Keywords} \textbf{
            - Linked Data, lightweight viewers, culling, BIM, SPARQL
        }
    
        The \ac{aec} industry has undergone significant technological transformations over the past few decades. The advent of 3D modeling constituted a major shift in the industry, heralding a new era of geometric data representation. The subsequent introduction of \ac{bim} has functioned as a versatile repository for semantics derived from various applications throughout the design and construction processes, including but not limited to cost estimation, energy analysis, and production planning. However, as \cite{Werbrouck2018} highlighted, the next challenge for the \ac{aec} industry is related to the domain-specific nature of current \ac{bim} software solutions, which remain inaccessible to other disciplines. This data \textbf{management} issue is currently being tackled by entities such as the \ac{lbd-cg} and academic institutions like the University of Ghent, leveraging the potential of Semantic Web technologies for more integrated and efficient data handling, thereby enabling interdisciplinary collaboration. The term \ac{ldbim} is used in this thesis to denote this emerging milestone.

        Alongside this evolution, in tandem with the increasing amount of semantic data, the geometric data describing a \ac{bim} model is also growing in size and complexity. This makes the task of visualization increasingly complex on newer, resource-limited devices used in the industry. Such devices, for example, smartphones and tablets, are becoming increasingly popular on construction sites. Existing solutions use a filtering step in the visualization process, known as culling. This computational step necessitates the processing of the entire 3D file or scene to cull it to the viewer's view frustum \parencite{Johansson2015}. This operation produces a minimal 3D scene, which is then utilized for the significantly more resource-intensive process of rendering.

        A bottleneck arises when the initial 3D scene, upon which the in-viewer culling process is performed, becomes too large for the device to handle. As a solution to this problem, this thesis proposes the use of a \textbf{pre-culling} step. In this process, and with the aid of Semantic Web technologies, a \ac{bim} model is stored and queried as a partitioned database. By segregating the storage of the \ac{bim} model and thus offloading the computationally intensive task of culling to a storage server in a preliminary culling step, the amount of data that needs to be processed by the visualization device is minimized.


        This thesis will therefore investigate the extent to which \ac{ldbim} geometry can be pre-culled. It will examine the minimum size of geometry that can be defined in a \ac{rdf} database. For this purpose, existing ontologies in the field of \ac{aec} or related fields such as \ac{gis} will be explored. These ontologies will also be scrutinized for their potential utilization within culling algorithms, similar to the work of \cite{Johansson2009}. In their paper, they exploited the semantics of a \ac{bim} model in an \ac{ifc} format to feed culling algorithms. As \ac{bim} models possess an inherent underlying hierarchy of \ac{bvh}, these are employed in modern in-viewer culling algorithms such as the \ac{chc}++ \parencite{Johansson2015}.

        \subsection*{Outline}
        \textsf{Chapter \ref{ch:introduction} -}
        This introduction chapter presents the motivation for this thesis and the research question. It also introduces the concept of culling and provides an overview of the structure of the thesis.

        \textsf{Chapter \ref{ch:linkedData} -}
        As an introduction to the Semantic Web, this chapter presents the Linked Data principles used in this thesis. It further explains complexity and size of \ac{bim} models within the Semantic Web databases that are \ac{rdf} graphs.

        \textsf{Chapter \ref{ch:stateOfTheArt} -}
        \cite{Johansson2015} pointed out that there is a scarcity of research exploring the performance of current Building Information Modeling (BIM) viewers. Thus, this state-of-the-art research aims to concentrate on the overall features of some promising newer viewers and the ontologies/tools
        that will be used in this thesis. 
        
        % Of the four researched ontologies is the \ac{bot} studied as it allows the description of underlying hierarchy of \ac{bim} models such as rooms and levels. Both the \ac{fog} and \ac{omg} ontologies are also proposed together to describe the geometry within the \ac{rdf} database, allowing the description of geometry in multiple formats and from multiple sources. As last, the GeoSPARQL ontology is presented for its ability to describe and query the spatial relationships between geometries. 

        \textsf{Chapter \ref{ch:dynamicQueries} -}
        The concept of dynamic queries is introduced in this chapter. It represents the automatic generation of queries responsible to feed culling algorithms for obtaining the data needed to visualise the scene visisble to the viewer. It therefore firstly develops on the needs of a \ac{ldbim} viewer, which lis at the base of the process. Secondly, three approaches are presented in which the culling action is performed: in the query, in the viewer and in situ. Each approach showcases the possibilities of culling by constructing \ac{sparql} queries.

        \textsf{Chapter \ref{ch:modularApproach} -}


        \textsf{Chapter \ref{ch:prototype} -}

        \textsf{Chapter \ref{ch:conclusion} -}

        \subsection*{Conclusion}
        This thesis has shown that culling \ac{ldbim} graphs to reduce the size of the scene a lightweight viewer has to manage in order to visualise a building model stored in a database using Semantic Web technologies is possible. It also presented multiple approaches to perform the culling, each with its own advantages and disadvantages, as well as a modular approach to implement the whole process in a web viewer. This technology has proven to be a viable solution to the demanding needs of the \ac{aec} industry when visualizing large \ac{bim} models. And it presents a strong foundation to expand upon in a diverse set of use-cases and scenarios in need of a 3D visual representation.
        \subsection*{References}
        \printbibliography

    \end{multicols}
\end{refsection}

\restoregeometry