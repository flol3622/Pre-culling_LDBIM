\selectlanguage{dutch}
\begin{center}
    \sffamily
    \huge Uitgebreid Abstract

    \Large Philippe Soubrier

    \normalsize
    Promotor: Prof.\ ir.-arch.\ Paulus Present\\
    Begeleiders: Ir.-arch.\ Jeroen Werbrouck, Prof.\ dr.\ ir.\ arch.\ Ruben Verstraeten
\end{center}
\begin{refsection}
    \defbibheading{bibliography}[]{}
    \begin{multicols}{2}
        \small
        \titlespacing\subsection{0pt}{5pt}{2pt}
        \emph{Abstract} \textbf{
            - Deze thesis pakt het probleem aan van het visualiseren van complexe \emph{\ac{bim}} modellen op apparaten met beperkte vermogen in de \emph{\ac{aec}} sector, waarbij \enquote{pre-culling} wordt geïntroduceerd om de 3D-scène te reduceren tot het zichtveld van de kijker. Door gebruik te maken van \emph{Semantic Web} technologieën zoals \emph{\ac{rdf}} en \emph{\ac{sparql}}, worden \ac{bim} modellen opgeslagen en opgevraagd als databases, wat hun segmentatie vergemakkelijkt.
        }

        \emph{Trefwoorden} \textbf{
            - Linked Data, minimale viewers, culling, BIM, SPARQL
        }

        De \ac{aec} industrie heeft de afgelopen decennia belangrijke technologische transformaties ondergaan. De komst van 3D-modellering heeft een grote verschuiving veroorzaakt in de industrie, waardoor een nieuw tijdperk van geometrische datavisualisatie is begonnen. De daaropvolgende introductie van \ac{bim} heeft gewerkt als een veelzijdige opslagplaats voor semantiek afkomstig van verschillende toepassingen gedurende het ontwerp- en bouwproces, inclusief maar niet beperkt tot: kostenraming, energieanalyse en \emph{\ac{fm}}. Zoals \cite{Werbrouck2018} benadrukte, ligt de volgende uitdaging voor de \ac{aec} industrie in de domeinspecifieke aard van de huidige \ac{bim} softwareoplossingen, die ontoegankelijk blijven voor andere disciplines. Dit \textbf{gegevensbeheerprobleem} wordt momenteel aangepakt door de \emph{\ac{lbd-cg}} en academische instellingen zoals de Universiteit van Gent, die het potentieel van \emph{Semantic Web}-technologieën benutten voor een meer geïntegreerde en efficiënte gegevensbehandeling, waardoor interdisciplinaire samenwerking mogelijk wordt. De term \emph{\ac{ldbim}} wordt in deze thesis gebruikt om dit opkomende gebied aan te duiden.

        Naast deze evolutie, en gelijktijdig met de toenemende hoeveelheid semantische gegevens, groeit de hoeveelheid geometrische data die een \ac{bim} model beschrijft in grootte en complexiteit. Dit maakt de taak van visualisatie steeds complexer op nieuwere apparaten met beperkte vermogen, die in de industrie worden gebruikt. Dergelijke apparaten, zoals smartphones en tablets, worden steeds populairder op bouwplaatsen. Bestaande oplossingen gebruiken een filteringstap in het visualisatieproces, bekend als \emph{culling}. Deze computationele stap vereist de verwerking van het volledige 3D-bestand of de scène om het te reduceren tot het zichtveld van de kijker \parencite{Johansson2015}. Deze operatie produceert een minimale 3D-scène, die vervolgens wordt gebruikt voor het aanzienlijk meer rekenintensieve renderproces.

        Er ontstaat een probleem wanneer de initiële 3D-scène, waarop het \emph{in-viewer culling} proces wordt uitgevoerd, te groot wordt voor het apparaat om te verwerken. Als oplossing voor dit probleem stelt deze thesis het gebruik van een \textbf{pre-culling} stap voor. In dit proces, met de hulp van \emph{Semantic Web} technologieën, wordt een \ac{bim} model opgeslagen en opgevraagd als een gepartitioneerde database. Door het opslaan van het \ac{bim} model te scheiden en de rekenintensieve taak van \emph{culling} te verplaatsen naar een opslagserver in een voorbereidende \emph{culling}-stap, wordt de hoeveelheid data die moet worden verwerkt door het visualisatieapparaat geminimaliseerd.

        Deze thesis zal onderzoeken in hoeverre de geometrie van \ac{ldbim} op voorhand kan worden \emph{pre-culled}. Het zal onderzoeken wat de minimale grootte van de geometrie kan zijn die kan worden gedefinieerd en opgezocht in een \ac{rdf} database. Voor dit doel zullen bestaande ontologieën van de \ac{aec} industrie of gerelateerde industrieën zoals \emph{\ac{gis}} worden verkend. Deze ontologieën zullen ook worden onderzocht voor hun potentiële gebruik binnen \emph{culling} algoritmen, vergelijkbaar met het werk van \cite{Johansson2009}. In hun artikel demonstreren ze het gebruik van de semantiek van een \ac{bim} model in een \emph{\ac{ifc}} formaat om \emph{culling} algoritmen te voeden. \cite{Johansson2015} hebben ook geconcludeerd dat de inherente onderliggende hiërarchie van \emph{\ac{bvh}} binnen \ac{bim} modellen gebruikt kan worden om de prestaties van in-viewer \emph{culling} algoritmen, zoals \emph{\ac{chc}++}, te verbeteren.

        \subsection*{Overzicht}
        \textsf{Hoofdstuk \ref{ch:introduction} -}
        Dit inleidende hoofdstuk presenteert de motivatie voor deze thesis en de onderzoeksvraagen. Het introduceert ook het concept van \emph{culling}.

        \textsf{Hoofdstuk \ref{ch:linkedData} -}
        Als inleiding tot het \emph{Semantic Web} presenteert dit hoofdstuk de \emph{Linked Data} principes die in deze thesis worden gebruikt. Verder legt het de complexiteit en grootte uit van \ac{bim} modellen binnen \emph{Semantic Web} databases.

        \textsf{Hoofdstuk \ref{ch:stateOfTheArt} -}
        \cite{Johansson2015} vermeld in zijn paper dat er een gebrek is aan onderzoek naar de prestaties van huidige \ac{bim} viewers. Daarom richt dit \emph{state-of-the-art} hoofdstuk zich op de algemene kenmerken van enkele veelbelovende nieuwe viewers en de ontologieën/tools die in deze thesis worden gebruikt.

        \textsf{Hoofdstuk \ref{ch:dynamicQueries} -}
        Het concept van dynamische \emph{queries} wordt in dit hoofdstuk geïntroduceerd. Deze representeren de automatische generatie van \emph{queries} die verantwoordelijk zijn voor het voeden van \emph{culling} algoritmen, om de data te verkrijgen die nodig is om de scène die voor de kijker zichtbaar is te visualiseren. Eerst worden de behoeften van een \ac{ldbim} viewer onderzocht. Ten tweede worden drie benaderingen gepresenteerd waar de \emph{culling} actie wordt uitgevoerd: in de \emph{query}, in de viewer, en in situ. Elke benadering toont \emph{culling} mogelijkheden aan door het construeren van \ac{sparql} \emph{queries}.

        \textsf{Hoofdstuk \ref{ch:modularApproach} -}
        Naast de daadwerkelijke \emph{culling}, stelt dit hoofdstuk een modulaire benadering voor om het \emph{culling} proces te implementeren in een \ac{ldbim} viewer. Deze robuuste basis dient als een \emph{framework} voor toekomstig onderzoek. Gebaseerd op het prototype van deze thesis, identificeert het \emph{framework} de verschillende componenten die nodig zijn om een \ac{ldbim} viewer te creëren en het \emph{culling} proces daarin te integreren. Het hoofdstuk ontleedt het \emph{framework} in vier hoofdcomponenten: de viewer zelf, de \emph{cache manager}, de \emph{query processor}, en de \emph{data fetcher}.

        \textsf{Hoofdstuk \ref{ch:prototype} -}
        Om de haalbaarheid van het voorgestelde \emph{culling} proces te demonstreren, wordt in dit hoofdstuk een prototype ontwikkeld. Dit prototype is een proof of concept dat de modulaire aanpak, die in het vorige hoofdstuk is gepresenteerd, implementeert binnen een \emph{web-based} viewer. Het toont de haalbaarheid van de voorgestelde dynamische \emph{queries} aan en demonstreert ook het vermogen tot uitbreiding, zoals de visualisatie van gerelateerde semantiek.

        \subsection*{Conclusie}
        \textsf{Hoofdstuk \ref{ch:conclusion} -} Deze thesis heeft aangetoond dat het haalbaar is om gebruik te maken van culling technieken op \ac{ldbim} \emph{graphs}. Dit proces vermindert effectief de scènegrootte die een lichtgewicht viewer moet verwerken voor het visualiseren van een groot gebouwmodel dat is opgeslagen in een database met behulp van \emph{Semantic Web} technologieën. Het presenteerde ook meerdere benaderingen om het snoeien uit te voeren, elk met zijn eigen voordelen en nadelen, evenals een modulaire benadering om het gehele proces in een \emph{web-based} viewer te implementeren. Deze technologie heeft bewezen een haalbare oplossing te zijn voor de veeleisende behoeften van de \ac{aec} industrie bij het visualiseren van grote \ac{bim} modellen. En het biedt een solide basis voor verdere ontwikkeling in diverse gebruiksscenario's en situaties die een 3D visuele representatie vereisen.

    \end{multicols}
    \subsection*{Referenties}
    \small
    {\renewcommand*{\bibfont}{\small}
        \printbibliography}

    \textsf{Demonstratie:} \url{https://github.com/flol3622/Pre-culling_LDBIM#demo}\\
    \textsf{Prototype:} \url{https://github.com/flol3622/LDBIM-viewer}
\end{refsection}