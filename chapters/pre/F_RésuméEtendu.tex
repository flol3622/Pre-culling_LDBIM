\selectlanguage{french}
\begin{center}
    \sffamily
    \huge Résumé étendu

    \Large Philippe Soubrier

    \normalsize
    Promoteur : Prof.\ Ing.-arch.\ Paulus Present\
    Superviseurs : Ing.-arch.\ Jeroen Werbrouck, Prof.\ dr.\ Ing.\ arch.\ Ruben Verstraeten
\end{center}
\begin{refsection}
    \defbibheading{bibliography}[]{}
    \begin{multicols}{2}
        \small
        \titlespacing\subsection{0pt}{5pt}{2pt}
        \emph{Résumé} \textbf{
            - Cette thèse aborde le problème de la visualisation de modèles \emph{\ac{bim}} complexes sur des appareils à capacités limitées dans le secteur de l'\emph{\ac{aec}}, en introduisant le \enquote{pre-culling} pour réduire la scène 3D au champ de vision de l'observateur. En utilisant des technologies du Web sémantique comme \emph{\ac{rdf}} et \emph{\ac{sparql}}, les modèles \ac{bim} sont stockés et demandés en tant que bases de données, ce qui facilite leur segmentation.
        }

        \emph{Mots clés} \textbf{
            - Linked Data, visualiseurs minimalistes, culling, BIM, SPARQL
        }

        L'industrie de l'\ac{aec} a subi d'importantes transformations technologiques ces dernières décennies. L'arrivée de la modélisation 3D a provoqué un grand changement dans l'industrie, marquant une nouvelle ère de représentation des données géométriques. L'introduction subséquente du \ac{bim} a servi de conteneur polyvalent pour la sémantique provenant de différentes applications tout au long du processus de conception et de construction, y compris mais sans s'y limiter, l'estimation des coûts, l'analyse énergétique et le \emph{\ac{fm}}. Cependant, comme \cite{Werbrouck2018} l'a souligné, le prochain défi pour l'industrie de l'\ac{aec} réside dans la nature spécifique du domaine des logicielles \ac{bim} actuelles, qui restent inaccessibles à d'autres disciplines. Ce problème de \textbf{gestion} des données est actuellement abordé par le \emph{\ac{lbd-cg}} et des institutions académiques comme l'Université de Gand, qui exploitent le potentiel des technologies du Web sémantique pour une gestion des données plus intégrée et efficace, permettant une collaboration interdisciplinaire. Le terme \ac{ldbim} est utilisé dans cette thèse pour désigner cette approche.

        Parallèlement à cette évolution, en tandem avec la quantité croissante de données sémantiques, les données géométriques décrivant un modèle \ac{bim} augmentent également en taille et en complexité. Cela rend la tâche de visualisation de plus en plus complexe sur les nouveaux appareils à ressources limitées utilisés dans l'industrie. De tels appareils, par exemple, les smartphones et les tablettes, deviennent de plus en plus populaires sur les chantiers de construction. Les solutions existantes utilisent une étape de filtrage dans le processus de visualisation, connue sous le nom de \emph{culling}. Cette étape de calcul nécessite le traitement de l'ensemble du fichier 3D ou de la scène pour la rogner jusqu'au champ de vision de l'observateur \parencite{Johansson2015}. Cette opération produit une scène 3D minimale, qui est ensuite utilisée pour le processus de rendu nettement plus gourmand en ressources.

        Un goulot d'étranglement se produit lorsque la scène 3D initiale, sur laquelle le processus de \emph{in-viewer culling} est effectué, devient trop volumineuse pour que l'appareil puisse la gérer. En solution à ce problème, cette thèse propose l'utilisation d'une étape de \textbf{pre-culling}. Dans ce processus, et avec l'aide des technologies du Web sémantique, un modèle \ac{bim} est stocké et interrogé comme une base de données partitionnée. En séparant le stockage du modèle \ac{bim} et donc en déchargeant la tâche intensive en ressources de \emph{culling} vers un serveur de stockage dans une étape préliminaire de \emph{culling}, la quantité de données qui doit être traitée par l'appareil de visualisation est minimisée.

        Cette thèse examinera donc dans quelle mesure la géométrie \ac{ldbim} peut être \emph{pre-culled}. Elle étudiera la taille minimale de la géométrie qui peut être définie et interrogée dans une base de données \ac{rdf}. À cette fin, les ontologies existantes dans le domaine de l'\ac{aec} ou des domaines connexes tels que l'\emph{\ac{gis}} seront explorées. Ces ontologies seront également examinées pour leur potentiel d'utilisation au sein des algorithmes de \emph{culling}, comme le montrent les travaux de \cite{Johansson2009}. Dans leur article, ils ont exploité la sémantique d'un modèle \ac{bim} au format \emph{\ac{ifc}} pour alimenter les algorithmes de \emph{culling}. La hiérarchie inhérente sous-jacente du \emph{\ac{bvh}} à l'intérieur des modèles \ac{bim} a démontré l'amélioration des performances des algorithmes de \emph{in-viewer culling}, par exemple, le \emph{\ac{chc}++}, comme cité par \parencite{Johansson2015}.

        \subsection*{Plan}
        \textsf{Chapitre \ref{ch:introduction} -}
        Ce chapitre d'introduction présente la motivation de cette thèse et les questions de recherche. Il introduit également le concept de \emph{culling} et donne un aperçu de la structure de la thèse.

        \textsf{Chapitre \ref{ch:linkedData} -}
        En guise d'introduction au Web sémantique, ce chapitre présente les principes du \emph{Linked Data} utilisés dans cette thèse. De plus, il explique la complexité et la taille des modèles \ac{bim} au sein des bases de données du Web sémantique.

        \textsf{Chapitre \ref{ch:stateOfTheArt} -}
        \cite{Johansson2015} a souligné qu'il y a une pénurie de recherches explorant les performances des visualiseurs \ac{bim} actuels. Ainsi, cette recherche sur l'état de l'art vise à se concentrer sur les caractéristiques globales de certains visualiseurs plus récents et prometteurs ainsi que sur les ontologies/outils qui seront utilisés dans cette thèse.

        \textsf{Chapitre \ref{ch:dynamicQueries} -}
        Le concept de requêtes dynamiques est introduit dans ce chapitre. Ce dernièr représente la génération automatique de requêtes responsables de l'alimentation des algorithmes de \emph{culling}, afin d'obtenir les données nécessaires pour visualiser la scène visible pour l'observateur. Par conséquent, tout d'abord, les besoins d'un visualiseur \ac{ldbim} sont explorés, ce trouvant à la base du processus. Deuxièmement, trois approches où l'action de \emph{culling} est effectuée sont présentées : dans la requête, dans le visualiseur, et sur site. Chaque approche met en évidence des possibilités de \emph{culling} en construisant des requêtes \ac{sparql}.

        \textsf{Chapitre \ref{ch:modularApproach} -}
        En plus du \emph{culling} proprement dit, ce chapitre propose une approche modulaire pour mettre en œuvre le processus de culling dans un visualiseur \ac{ldbim}. Cette base solide sert de \emph{framework} pour des recherches futures. Informé par le prototype de cette thèse, il identifie les différents composants nécessaires pour créer un visualiseur \ac{ldbim} et intégrer le processus de \emph{culling} en son sein. Il dissèque le \emph{framework} en quatre composants principaux : le visualiseur lui-même, le gestionnaire de cache, le processeur de requêtes, et le récupérateur de données.

        \textsf{Chapitre \ref{ch:prototype} -}
        Pour démontrer la faisabilité du processus de \emph{culling} proposé, un prototype est développé dans ce chapitre. Ce prototype est une preuve de concept qui met en œuvre l'approche modulaire présentée dans le chapitre précédent au sein d'un visualiseur basé sur le web. Il démontre la faisabilité des requêtes dynamiques proposées et montre également la capacité d'expansion, spécifiquement la visualisation de sémantique liée.

        \subsection*{Conclusion}
        \textsf{Chapitre \ref{ch:conclusion} -} Cette thèse a démontré la faisabilité d'utiliser des filtres sur les graphes \ac{ldbim}. Ce processus réduit efficacement la taille de la scène qu'un visualiseur léger doit gérer pour visualiser un grand modèle de bâtiment stocké dans une base de données en utilisant des technologies du Web Sémantique. Elle a également présenté plusieurs approches pour effectuer le \emph{culling}, chacune avec ses propres avantages et inconvénients, ainsi qu'une approche modulaire pour mettre en œuvre l'ensemble du processus dans un visualiseur web. Cette technologie s'est avérée être une solution viable pour répondre aux besoins exigeants de l'industrie de l'\ac{aec} lors de la visualisation de grands modèles \ac{bim}. Et elle présente une base solide pour être développée dans le cas d'utilisations et de scénarios diversifiés, nécessitant une représentation visuelle 3D.
    \end{multicols}
    \subsection*{Références}
    \small
    {\renewcommand*{\bibfont}{\small}
        \printbibliography}

    \textsf{Démonstration :} \url{https://github.com/flol3622/Pre-culling_LDBIM#demo}\\
    \textsf{Prototype :} \url{https://github.com/flol3622/LDBIM-viewer}
\end{refsection}