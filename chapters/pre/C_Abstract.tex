\begin{center}
    \sffamily
    \huge Pre-culling geometric linked building data
    for lightweight viewers

    \Large Philippe Soubrier

    \normalsize
    Supervisor: Prof.\ ir.-arch.\ Paulus Present         \\
    Counselors: Ir.-arch.\ Jeroen Werbrouck \\
    Prof.\ dr.\ ir.\ arch.\ Ruben Verstraeten
\end{center}

Master’s dissertation submitted in order to obtain the academic degree of \\
Master of Science in de ingenieurswetenschappen: architectuur

Academic year 2022-2023
\section*{\LARGE Abstract}
A significant interaction with a building's digital twin in the \ac{aec} sector is the 3D visualisation of a \ac{bim} model. As the industry evolves in this digital era, the increasing amount of digital data exponentially amplifies the complexity of a building's semantic and geometrical definition. This geometric complexity poses a challenge for the visualisation of \ac{bim} models on newer, resource-limited devices used in the industry, such as smartphones and tablets.

Proposing a solution to this issue, this thesis introduces the concept of initial filtering, or pre-culling in computer graphics terminology, the geometrical data of a \ac{bim} model on the device where it is stored before transmitting it to the resource-limited device. Thus leaving this device with a minimal, relevant 3D scene confined to its view frustum.

The focus of this thesis is placed on the utilization of Semantic Web technologies,such as the \ac{rdf} and the \ac{sparql}, to store and query \ac{bim} models as databases. This enables the datasource to be divided into smaller, more manageable chunks containing only the data required for the visualisation purpose.

This concept is brought to life through the proposal of a web-based 3D viewer prototype together with its modular and extendable structure. This enables not only the demonstration of the pre-culling concept, but also establishes a robust groundwork for further research and alternative implementations.
                   
\vfill
\emph{Keywords} \textbf{
    - Linked Data, lightweight viewers, culling, BIM, SPARQL
}