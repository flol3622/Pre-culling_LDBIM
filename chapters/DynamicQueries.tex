\chapter{Dynamic Queries}
% structure of this chapter:
% main = dynamic queries
% Explain:
% - what I mean by that
% - basic requirements = viewer
% - beauty = can be extendend, 
%   more filetypes, not only geometry
% - 3 possible implementations, 
%   computation always in different place
% - making use of EXISTING DATA,
%   can be extended to new dedicated ontologies

\section{Viewer}
% - importance of viewer
% - refer to state of the art existing viewers
\subsection{Requirements}
% - different file formats, different sources.
% - reusing first diagram
% - why XEOKIT sdk (practical)

\subsection{Beyong geometry}
% - possibilities of viewer extends classical approach

\subsubsection{BCF integration}
% - BCF viewpoints in xeokit
% - what is bcf...

\subsubsection{Visualising semantic}
% - element associated data
% - physical: thermal, acoustic, ...
% - non-physical: cost, time, pothologies, ...
% - all  with temporal dimension / history

\section{In situ WKT location}
% - limited to 2D
% - viewer: knows location, AR: knnows 
% - on building site location system 
%   / BIM model translated into WKT literals in model
% - use of WKT and GeoSPARQL
% - to which extends needs every element a location
% - which queries are possible


\section{In viewer bot:Space indentification}
% - 1 database + 1 3D enabled instance (viewer) 
%   => viewer can raytrace identify wright space (little computation)
% - 2 layered viewer visible elements / invisible spaces
% - sequence diagram
%   - manual first step + adjacent spaces
%   - all spaces
%   - combination of spaces
% - sample queries

\section{In query OBJ geometry filtering}
% - no geometric but string analysis in query
% - AABBox filtering of for example obj rooms
% - all negative points


