\chapter{State of the art}
-> Introduction to the chapter, both tools and existing approaches to create overview of the state of the art

\section{Related Technologies and Tools}
\subsection{Qoniq and LOD Streaming for BIM}
-> Why I chose this one (why special)\\
-> LOD streaming principle\\
-> Probably present it as a goal but in the older framework (for effectiveness(esthetics and performance))\\
->In unity, the maximum amount of ram cannot exceed 2Gb\footcite{UnityWebGL}

\subsubsection{Qoniq's approach to LOD streaming}
(T. Strobbe, personal communication, November 25, 2022)\\
-> in contrary to \cite{Johansson2015}

\subsubsection{Advantages and challenges}

\subsubsection{Potential applicability to LDBIM}
-> as stated before advantage of LOD existing

\subsection{ld-bim.web.app}
\url{https://ld-bim.web.app/}
-> Where does it come from\\
-> Detailed explanation of the features / capabilities\\
-> Detailed fragmentation of missed opportunities / how this thesis positions itself to it

\subsection{\acs{aec} related ontologies}
-> as proposed in \cite{Johansson2015}\\
--> using BIM semantics to introduce new culling Techniques (5)\\
--> CHC++ \\
-> Present the following as related work but still in early stage, needs way more research (for computer scientists)

\subsubsection{\acs{bot}}
\subsubsection{\acs{fog} and \acs{omg}}
\subsection{\acs{gis} related ontologies}
-> Highlighting maturity and usefull data

\subsubsection{geoSPARQL}
-> 2D Limitations

\section{Existing Approaches in BIM 3D Viewers and Visualization Techniques}
\begin{itemize}
    \item DDS CAD Viewer
    \item Tekla BIMsight
    \item Autodesk Navisworks
    \item Solibri Model Viewer
\end{itemize}

\subsection{General Features}
-> \cite{Johansson2015}\\
--> table 3 about Acceleration techniques\\

\subsection{Interoperability}
% 1. Importance of supporting multiple file formats.
% 2. Data exchange standards in BIM.
% 3. Integrating BIM viewers with other tools.

\subsection{Scalability}
% 1. Performance challenges with large-scale BIM models.
% 2. Memory usage and real-time rendering.
% 3. Strategies for optimizing viewer performance.

\subsection{Collaboration and Data Sharing}
% 1. BIM viewers' role in project collaboration.
% 2. Data sharing and communication among stakeholders.
% 3. Synchronization and version control features.

\subsection{Customization and Extensibility}
% 1. APIs and plugins for viewer customization.
% 2. Adapting viewers to specific project requirements.
% 3. Integrating BIM viewers with other systems.

