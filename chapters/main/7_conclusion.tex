\chapter{Conclusion and future work} \label{ch:conclusion}
This thesis has explored the feasibility of culling geometry from a \ac{ldbim} graph for streaming to lightweight viewers. The goal was to minimize the volume of data within the viewer itself, as existing in-viewer culling techniques encounter processing limitations when the scene continues to expand in size, particularly when managing non-active parts of the scene. Two primary questions were therefore posed:

\begin{enumerate}
    \item To what extent can \ac{ldbim} geometry be culled to be streamed to lightweight viewers? (Section \ref{subsec:rq1})
    \item Can existing semantics and ontologies be utilized to inform potential culling algorithms? (Section \ref{subsec:rq2})
\end{enumerate}

Both questions were initially explored by reviewing the current state of the art in the field of Semantic Web and culling algorithms. This evolved into a hands-on approach, where the development of a prototype led to new insights.

\section{Object-level Culling}
The usage of existing ontologies, like \ac{fog} and \ac{omg}, has facilitated the definition of geometry at the object level, essentially pinpointing the smallest units that are subject to culling. As a result, this thesis did not address culling techniques such as back-face culling, as these are handled by the viewer itself, not the database. The \ac{fog} and \ac{omg} ontologies, although not specifically explored in this thesis, also allow the linking of auxiliary geometry files, such as texture maps, to the entities. Future research could explore the possibilities of implementing the auxiliary geometry data in the viewer, utilizing the \ac{omg} level 2 data pattern, thereby enhancing the visual quality of the resulting scene.

\section{Culling Algorithms}
Three culling algorithms were proposed, each performing culling operations in radically different ways, showcasing the possibilities of culling by constructing \ac{sparql} queries.

The first algorithm (Section \ref{sec:inSituWKT}) utilizes the \ac{wkt} serialization of \ac{bot} rooms to evaluate GeoSPARQL functions. This approach highlighted the lack of 3D operations in GeoSPARQL, indicating a crucial need in the \ac{aec} industry for a standard for 3D spatial operations in \ac{sparql}, which, as of now, does not exist.

The second algorithm (Section \ref{sec:inViewer}) leveraged the viewer's 3D engine capabilities to perform 3D operations not feasible with the first algorithm. It utilized ray tracing to determine the room in which the observer was located. Although this algorithm offloaded 3D operations from the database to the viewer, which appears counter-intuitive to the thesis's objective, the operations were optimized to use as little computing power as possible by performing the ray tracing only on a subset of rooms in the scene.

The third algorithm (Section \ref{sec:inQuery}) proposes a method to implement 3D operations in the form of string operations within the \ac{sparql} query itself. As \ac{sparql} string operations are limited, a custom JavaScript function was added to the \ac{sparql} endpoint. However, the implementation is specific to the endpoint used in this thesis, GraphDB, as no standard exists for custom functions in \ac{sparql} endpoints. The developed function is also limited to the analysis of the OBJ geometry format, but similar functions could be developed for other formats.

All options utilized the \ac{bot} ontology to zoom out from the object-level to cull at the room-level, leveraging the inherent underlying hierarchy of \ac{bim} models which is described in the graph using the \ac{bot} ontology. It was found to produce coherent results and reduce the computational resources needed for the culling. However, this hierarchy is only relevant when examining the interior of the building and becomes irrelevant when viewing the exterior of the building. Further research could explore the potential of culling algorithms for scenarios outside the building.

\section{Modular approach and prototype}
The modular approach (Chapter \ref{ch:modularApproach}) presented in this thesis' prototype (Chapter \ref{ch:prototype}) was found to be successful in showcasing the feasibility of culling geometry from a \ac{ldbim} graph. It also highlighted the need for further research into each step of the process, from the cache management to the moment when the query has to run again. However, it does showcase a strong foundation for future development and research.

To achieve the loading of different sources and formats, the existing ontologies \ac{fog} and \ac{omg} were used. However, these were stretched to their limits, as newer ontologies and semantics are needed to fully leverage the potential and versatility of Linked Data. The developed prototype highlighted the need for additional data about each geometry description, such as orientation and scale, which can differ between files and formats.

\section{Database}
The database used in this thesis was based on the work of Mads Holten Rasmussen, which was found to be relatively easy to adapt to the needs of this thesis, as it already contained geometrical (obj geometry), spatial (\ac{wkt} serialisation) and relational (\ac{bot} relations) data. Despite its usefulness, the limited size prevented the evaluation of the performance of the culling algorithms on larger datasets. The lack of larger datasets was also found to be a general problem in the field of \ac{ldbim}, as no other was found during this research. Besides its size, the dataset was also found to be missing some relations, such as the adjacent spaces and the floors associated with the rooms. These relations are important for the culling algorithms as they are used to zoom out from the object-level to the room-level.

\section{Conclusion}
This thesis has demonstrated the feasibility of using culling techniques on \ac{ldbim} graphs. This process effectively reduces the scene size that a lightweight viewer must handle for visualising a large building model stored in a database using Semantic Web technologies. It also presented multiple approaches to perform the culling, each with its own advantages and disadvantages, as well as a modular approach to implement the whole process in a web viewer. This technology has proven to be a viable solution to the demanding needs of the \ac{aec} industry when visualizing large \ac{bim} models. And it provides a solid foundation for further development in various use cases and scenarios requiring 3D visual representation.